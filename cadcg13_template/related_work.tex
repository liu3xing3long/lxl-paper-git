
Our work relates to vessel extraction from image and vessel
reconstruction etc. We briefly review them in the following
categories.

\subsection*{Vessel Extraction}
Hoover et al.~\cite{Hoover} use a mathematical filter to entails a
broad range of vessel enhancement and Li et al.~\cite{Li} do this
using a non-linear filter. Frangi et al.~\cite{Frangi} use the eigen
values of Hessian matrix to extract the tube-like structures from the
X-Ray images. Condurache et al.~\cite{Condurache} use this method
while adding a hysteresis thresholding method to purify the extracted
data.

\subsection*{Centerline Extraction}
Centerline extraction consists of six kinds of techniques: pattern
recognition techniques, deformable model based techniques
\cite{Deformable1}\cite{Deformable2}\cite{Deformable3}, tracking-based
techniques \cite{Tracking1}\cite{Tracking2}\cite{Tracking3} \cite{Tracking4}
\cite{Tracking5}, artificial intelligence-based techniques, neural
network-based techniques and miscellaneous tube-like object detection
techniques. Each one contains many sub-types such as multi-scale approaches,
mathematical morphology approaches. Readers please refer to~\cite{Kirbas}
for an overview of the centerline extraction technologies.

\subsection*{3D Reconstruction}

As with 3D reconstruction, Wellnhofer et al.~\cite{Wellnhofer} and Messanger
et al.~\cite{Messenger} evaluate that 3D reconstructions of coronary
arteries from 2D X-Ray image sequences permit accurate results of the real
data. The two types of the X-Ray machine lead to two slightly different ways
of 3D reconstruction. The biplane system takes two (mostly) synchronized
projection of the coronary arteries~\cite{Wellnhofer}\cite{Messenger}.
Meanwhile the mono-plane (single-plane) system~\cite{Gollapudi} can just
take one view at the same time, therefore selection of asynchronous images
from multiple views is needed. However, using only two 2D projections to
reconstruct the complex 3D topology of coronary artery is often not
sufficient. Movassaghi et al.~\cite{Movassaghi} uses multiple projections
for realistic vessel lumen simulations, but only uses two for 3D centerline
reconstruction. Sprague et al.~\cite{Sprague} utilizes the benefits of three
projections experimentally. Hansis et al.~\cite{Hansis} has used multiple
projections from a single rotational X-Ray angiography to reconstruct the 3D
centerline and the topology. Nguyen et al.~\cite{Nguyen} propose a method
based on motion and multiple views using a single-plane imaging system. They
only consider the rotation and scaling of the heart motion but rigorously
the heart motion during contraction and relaxation consists of five
movements: translation, rotation, wringing, accordion-like motion and
movement towards the center of the ventricular chamber~\cite{Marcus}.
Therefore, a simple motion can not generally describe the heart cardiac
cycle.

Other routines such as knowledged-based or rule-based have been
proposed for 3D reconstruction using the vascular network
model~\cite{rule_based1} \cite{rule_based2}. Since their rules or
knowledge are designed for specific conditions, it is not easy to
generalize these kinds of methods to process artery data.

In~\cite{opti_est1}\cite{opti_est2}\cite{opti_est3}, optimal
estimation are investigated with a two-step approach based on
maximum-likelihood and minimum-variance estimation. They use a linear
algorithm to compute the preliminary estimates as the initial
estimates for the process of optimal estimation. Due to the huge
computation, none of the existing techniques have been used in
clinical therapy.

\subsection*{Other Focuses}

Another focus on 3D reconstruction is on the elimination or
minimization of foreshortening and overlap of the coronary arteries
which is a prerequisite for an accurate quantitative coronary analysis
such as the vessel lengths and aneurysms. In~\cite{opti_view1},
they focus on the minimization of vessel foreshortening relative to
a single arterial segment. Sato et al.~\cite{opti_view2} and
Finet et al.~\cite{opti_view3} introduce an
optimal view selection method considering both foreshortening and
vessel overlap. Chen et al.~\cite{James_Chen} use bifurcation points
and the vessel directional vectors of bifurcations to register between image
pairs. Meanwhile, they also proposed a method of selecting
the minimized foreshortening views. But, it depends on bifurcations
overmuch and requires at least five pairs of bifurcations to ensure
accurate transformation. Also, their work is time consuming with a
whole procedure of ten minutes.
